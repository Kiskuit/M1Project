%-------------------------------------------------------------------------
%
% Master-2 NPAC 
%
% NPAC Lab Projects: template article
%
% The final article should not have more than 4 pages
% you are not allowed to modify fonts, page size and margins.
% Anything after the end of the 4th page will be ignored.
%
%-------------------------------------------------------------------------
%
% To compile your article with LaTeX, do this on the terminal:
%   
%        pdflatex article-template.tex
%
% This will produce a PDF file. Figures may be PDF files, images, etc.
% If you have PS/EPS figures convert them first in PDF files.
%
% You may need to compile twice to get the references properly set.
% 
%-------------------------------------------------------------------------
% 
%
\documentclass[final,12pt]{article}
%
\usepackage{npac}
\usepackage{amsmath}
\usepackage{url}
\usepackage{graphicx}
\usepackage{color}
\usepackage[T1]{fontenc}
\usepackage[french]{babel}
\usepackage[utf8]{inputenc}

%
\begin{document}
%
\title{Manipulation de suites P-récursives avec SageMath}
\author{Mathis \textsc{Carisntan} \& Aurélien \textsc{Lamoureux} \\ {\small sous la responsabilité de Marc \textsc{Mezzarobba}}}
%
\date{09/03/2017}

\maketitle
%
\begin{abstract}
    Ce rapport présente le travail que nous avons effectué au cours de ce projet.
    Nous présentons dans un premier temps ce que sont les suites P-récursives,
    ainsi que l'outil SageMath. Puis nous expliquons les motivations de ce projet.
    Enfin, nous détaillons les choix et détails de l'implémentation que nous avons réalisé,
    avant de discuter des limites de celle-ci et des possibles améliorations.
\end{abstract}

\section{Introduction}
    \label{sec:intro}
    Nous nous intéressons ici aux suites p-récursives, et comment les manipuler dans SageMath (ou Sage). \\
    {\color{red} ...TODO...}
    \subsection{Suites p-récursives \& Algèbre d'Ore}
        (Motivations : Repr exactes de suites, utilisée dans différents domaines des maths/sciences)
        \label{ssec:prec}
        {\color{red} ...TODO...}
    \subsection{Python \& Sage}
        \label{ssec:sage}
        {\color{red} ...TODO...}

\section{Méthodologie de travail, et progression}
    \label{sec:methodo}
    La première tâche à laquelle nous nous sommes attelés, a été de chercher à comprendre notre sujet
    (les suites p-récursives) et nos outils (Python et Sage). Une fois cette étape effectuée, nous avons
    commencé à discuter de l'implémentation. Nous nous sommes rapidement mis d'accord avec notre encadrant,
    qu'il était plus pertinent d'un point de vue pédagogique de d'abord créer un module python,
    utilisant les fonctionnalités de Sage. Puis, une fois ce module eprouvé, le réécrire 
    en utilisant la syntaxe de Sage. Cette manière de procéder nous a permis de nous concentrer initialement
    sur le fond, et non la forme, puisque nous étions plus familier avec Python.
    \subsection{Module Python}
        \label{ssec:py}
        La base du module a été d'écrire une classe Python ({\color{red} init. n'étend aucun classes}). Cette classe devait notamment permettre
        d'utiliser la représentation basée sur la relation de récurrence, et des conditions initiales.
        Immédiatement après, nous avons surchargé l'opérateur \texttt{\_\_getitem\_\_} pour accéder
        au n-ième terme de la suite. Initialement, nous calculions tous les termes de la suite, jusqu'à 
        celui voulu, que nous renvoyions, mais cette méthode est très inefficace. Nous avons donc résolu
        d'utiliser la fonction \texttt{forward\_matrix} du module \textsc{ore\_algebra} à la place ({\color{red} exemple
        et comparaison complex avec Fibo?}) 
        \\{\color{red}calculer ts les elts vs calculer que le bon élement})\\
        Par la suite, nous avons également surcharger les opérateurs d'addition, soustraction et multiplication, 
        en accord avec les lois de l'agèbre d'Ore.




% TODO bibtex? zotero?
\begin{thebibliography}{}
\bibitem{Boh}A. Bohr and B.R. Mottelson, Nuclear Structure, vol. 2, Benjamin,
New York, 1975.

\bibitem{ipn-web} \url{http://ipnweb.in2p3.fr}

\bibitem{trip} Nick Park, \textsl{A Grand Day Out}, 1989, \url{http://en.wikipedia.org/wiki/A_Grand_Day_Out}

\end{thebibliography}


\end{document}
