%-------------------------------------------------------------------------
%
% Master-2 NPAC 
%
% NPAC Lab Projects: template article
%
% The final article should not have more than 4 pages
% you are not allowed to modify fonts, page size and margins.
% Anything after the end of the 4th page will be ignored.
%
%-------------------------------------------------------------------------
%
% To compile your article with LaTeX, do this on the terminal:
%   
%        pdflatex article-template.tex
%
% This will produce a PDF file. Figures may be PDF files, images, etc.
% If you have PS/EPS figures convert them first in PDF files.
%
% You may need to compile twice to get the references properly set.
% 
%-------------------------------------------------------------------------
% 
%
\documentclass[final,12pt]{article}
%
\usepackage{npac}
\usepackage{amsmath}
\usepackage{amsfonts}
\usepackage{amssymb}
\usepackage{url}
\usepackage{graphicx}
\usepackage{color}
\usepackage[T1]{fontenc}
\usepackage[french]{babel}
\usepackage[utf8]{inputenc}

%
\begin{document}
%
\title{Manipulation de suites P-récursives avec SageMath}
\author{Mathis \textsc{Carisntan} \& Aurélien \textsc{Lamoureux} \\ {\small sous la responsabilité de Marc \textsc{Mezzarobba}}}
%
\date{09/03/2017}

\maketitle
%
\begin{abstract}
    Ce rapport présente le travail que nous avons effectué au cours de ce projet.
    Nous présentons dans un premier temps ce que sont les suites P-récursives,
    ainsi que l'outil SageMath. Puis nous expliquons les motivations de ce projet.
    Enfin, nous détaillons les choix et détails de l'implémentation que nous avons réalisé,
    avant de discuter des limites de celle-ci et des possibles améliorations.
\end{abstract}

\section{Introduction}
    \label{sec:intro}
    {\color{red} ...TODO...}
    \subsection{Suites p-récursives \& Algèbre d'Ore}
        \label{ssec:prec}
        \par Les suites sont beaucoup utilisées en mathématiques et dans différents domaines
        scientifiques, et on cherche, comme souvent en informatique, à en avoir une
        représentation exacte. De plus, il est généralement important que cette représentation
        soit également efficace pour la manipulation mathématique de ces suites.
        \par On s'intéresse ici en particulier aux suites dites p-récursives.
        Une suite $(u_n)_{n\in\mathbb N}$ sur un corps $\mathbb K$ est dite p-récursive
        si elle est solution d'une équation de la forme :
        \begin{equation}
            \sum_{i=0}^s p_i(n) u_{n+i} = 0
        \end{equation}
        où, les $p_i$ sont des polynômes en n. Il est importante de noter que contrairement
        à des suites arbitraires, les suites p-récursives, bien que comportant un nombre
        potentiellement infini de termes, peuvent être représentées exactement simplement
        avec la relation de récurrence, et les conditions initiales.
        Des exemples communs de suites p-récursives
        sont par exemple la suite de Fibonacci, ou la fonction factorielle.
        \begin{eqnarray*}
            \textnormal{Fibonacci : } u_{n+2} - u_{n+1} - u_n = 0, \quad F_0=0, F_1=1\\
            \textnormal{Factorielle : } u_{n+1} - (n+1)*u_n = 0, \quad 0!=0
        \end{eqnarray*}
        De plus, les suites p-récursives forment un aneau ({\color{red} Donner les détails?}).
        Dès lors, il semble pertinent
        de réaliser une implémentation utilisant ces propriétés mathématiques afin de
        manipuler et utiliser les suites p-récursives.\\
        {\color{red} TODO : ore\_algebra what \& why}
    \subsection{Python \& Sage}
        \label{ssec:sage}
        \par Sage est un outil de calcul formel libre.
        Il a été créé notamment pour proposer
        une alternative \textit{opensource} aux logiciels existants comme Mathematica,
        Matlab, Maple \ldots Contrairement à ces logiciels, Sage s'appuie sur des outils
        et librairies déjà existants comme NumPy, SciPy, matplotlib, FLINT et d'autres...
        L'utilisation de ces outils et unifiée et uniformisée au travers d'un langage
        basé sur Python. Ce langage présente une syntaxe qui diffère légérement de celle
        de Python. Ainsi, Sage est doté d'un "pré-analyseur",
        qui transforme les idiomes Sage en pur Python.
        Ainsi, il est possible d'écrire des librairies pour en Python pur ou en "langage sage".
        Bien qu'il existe également d'autres méthodes, on ne s'est intéressé qu'à celles-ci au
        cours du projet.
        \par Comme évoqué plus haut, Sage est basé sur Python, et c'est donc naturellement
        que nous avons choisi ce langage pour le projet.
        En particulier, Python 2, puisque Sage n'est pas compatible avec Python 3
        (bien que des efforts soient faits en ce sens).
        \par Bien que Sage fournisse de nombreuses librairies mathématiques,
        il n'inclut pas encore officiellement de librairie pour l'algèbre d'Ore.
        Nous avons eu donc recours à une bibliothèque en cours de développement par
        la communauté qui implémente l'algèbre d'Ore. % TODO ref papier/site
\section{Méthodologie de travail, et progression}
    \label{sec:methodo}
    \par La première tâche à laquelle nous nous sommes attelés a été une recherche bibliographique,
    pour comprendre le sujet (les suites p-récursives), et nos outils (Sage et Python).
    Les résultats de cette démarche sont présentés dans la partie \ref{sec:intro}.\\
    Puis nous avons commencé à discuter de l'implémentation. En accord avec notre encadrant,
    nous avons estimé qu'il était plus pertinent d'un point de vue pédagogique, de réaliser
    d'abord un module Python. Puis, une fois ce module eprouvé, si nous avions le temps,
    le réécrire avec la syntaxe de Sage. Cette manière de procédé devait permettre de se
    concentrer initialement sur le fond, et non la forme, puisque nous étions plus familier
    avec Python. {\color{blue} Il s'est avéré par la suite que nous n'avons pas eu le temps
    d'aborder la réécriture}.
    \subsection{Module Python}
        \label{ssec:py}
        La base du module a été d'écrire une classe Python ({\color{red} init. n'étend aucun classes}). Cette classe devait notamment permettre
        d'utiliser la représentation basée sur la relation de récurrence, et des conditions initiales.
        Immédiatement après, nous avons surchargé l'opérateur \texttt{\_\_getitem\_\_} pour accéder
        au n-ième terme de la suite. Initialement, nous calculions tous les termes de la suite, jusqu'à 
        celui voulu, que nous renvoyions, mais cette méthode est très inefficace. Nous avons donc résolu
        d'utiliser la fonction \texttt{forward\_matrix} du module \textsc{ore\_algebra} à la place ({\color{red} exemple
        et comparaison complex avec Fibo?}) 
        \\{\color{red}calculer ts les elts vs calculer que le bon élement, exemple de différence tps exec sur 100000!})\\
        Par la suite, nous avons également surcharger les opérateurs d'addition, soustraction et multiplication, 
        en accord avec les lois de l'agèbre d'Ore.




% TODO bibtex? zotero?
\begin{thebibliography}{}
\bibitem{Boh}A. Bohr and B.R. Mottelson, Nuclear Structure, vol. 2, Benjamin,
New York, 1975.

\bibitem{ipn-web} \url{http://ipnweb.in2p3.fr}

\bibitem{trip} Nick Park, \textsl{A Grand Day Out}, 1989, \url{http://en.wikipedia.org/wiki/A_Grand_Day_Out}

\end{thebibliography}


\end{document}
