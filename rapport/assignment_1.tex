%----------------------------------------------------------------------------------------
%	PACKAGES AND OTHER DOCUMENT CONFIGURATIONS
%----------------------------------------------------------------------------------------

\documentclass[12pt]{article}

\usepackage{fancyhdr} % Required for custom headers
\usepackage{lastpage} % Required to determine the last page for the footer
\usepackage{extramarks} % Required for headers and footers
\usepackage{graphicx} % Required to insert images
\usepackage{lipsum} % Used for inserting dummy 'Lorem ipsum' text into the template
\usepackage{amsmath}
\usepackage{amsfonts}
\usepackage{amssymb}
\usepackage{vmargin}
\usepackage{url}
\usepackage{color}
\usepackage{calc}
\usepackage[T1]{fontenc}
\usepackage[french]{babel}
\usepackage[utf8]{inputenc}

% Margins
%\topmargin=-0.45in
%\evensidemargin=0in
%\oddsidemargin=0in
%\textwidth=6.5in
%\textheight=9.0in
%\headsep=0.25in 
\setmargrb{3cm}{2cm}{3cm}{2cm}

\linespread{1.1} % Line spacing

% Set up the header and footer
\pagestyle{fancy}
\lhead{\hmwkAuthorName} % Top left header TODO care with this
\cfoot{} % Bottom center footer
\rfoot{Page\ \thepage\ of\ \pageref{LastPage}} % Bottom right footer
\renewcommand\headrulewidth{0.4pt} % Size of the header rule
\renewcommand\footrulewidth{0.4pt} % Size of the footer rule

%\setlength\parindent{0pt} % Removes all indentation from paragraphs

%----------------------------------------------------------------------------------------
%	DOCUMENT STRUCTURE COMMANDS
%----------------------------------------------------------------------------------------

%\setcounter{secnumdepth}{0} % Removes default section numbers

%----------------------------------------------------------------------------------------
%	NAME AND CLASS SECTION
%----------------------------------------------------------------------------------------

\newcommand{\hmwkTitle}{Manipulation de suites P-récursives avec SageMath} % Assignment title
\newcommand{\hmwkDueDate}{\today} % Due date
\newcommand{\hmwkClass}{Projet SFPN} % Course/class
\newcommand{\hmwkClassTime}{} % Class/lecture time
\newcommand{\hmwkClassInstructor}{Marc \textsc{Mezzarobba}} % Teacher/lecturer
\newcommand{\hmwkAuthorName}{Mathis \textsc{Caristan} \& Aurélien \textsc{Lamoureux}} % Your name

%----------------------------------------------------------------------------------------
%	Our commands
%----------------------------------------------------------------------------------------

\newlength{\charwidth}
\setlength{\charwidth}{\widthof{\S}}
\newcommand{\uline}{\underline{\hspace{2\charwidth}}}

%----------------------------------------------------------------------------------------
%	TITLE PAGE
%----------------------------------------------------------------------------------------

\title{
\vspace{1in}
\textmd{\textbf{\hmwkClass:\ \hmwkTitle}}\\
\vspace{0.1in}\large{Encadré par\ \hmwkClassInstructor}
}

\author{\textbf{\hmwkAuthorName}}
\date{\today} % Insert date here if you want it to appear below your name

%----------------------------------------------------------------------------------------

\begin{document}

\maketitle
\begin{abstract}
    Ce rapport présente le travail que nous avons effectué au cours de ce projet.
    Nous présentons dans un premier temps ce que sont les suites P-récursives,
    ainsi que l'outil SageMath. Puis nous expliquons les motivations de ce projet.
    Enfin, nous détaillons les choix et détails de l'implémentation que nous avons réalisé,
    avant de discuter des limites de celle-ci et des possibles améliorations.
\end{abstract}
% TODO logo UPMC & stuff

%----------------------------------------------------------------------------------------
%	TABLE OF CONTENTS
%----------------------------------------------------------------------------------------

\setcounter{tocdepth}{2} % Uncomment this line if you don't want subsections listed in the ToC

\newpage
\tableofcontents
\newpage

%----------------------------------------------------------------------------------------
%	PROBLEM 1
%----------------------------------------------------------------------------------------

% To have just one problem per page, simply put a \clearpage after each problem

\section{Introduction}
    \label{sec:intro}
    {\color{red} ...TODO...}
    \subsection{Suites p-récursives \& Algèbre d'Ore}
        \label{ssec:prec}
        \par Les suites sont beaucoup utilisées en mathématiques et dans différents domaines
        scientifiques, et on cherche, comme souvent en informatique, à en avoir une
        représentation exacte. De plus, il est généralement important que cette représentation
        soit également efficace pour la manipulation mathématique de ces suites.
        \par On s'intéresse ici en particulier aux suites dites p-récursives.
        Une suite $(u_n)_{n\in\mathbb N}$ sur un corps $\mathbb K$ est dite p-récursive
        si elle est solution d'une équation de la forme :
        \begin{equation}
            \sum_{i=0}^s p_i(n) u_{n+i} = 0
        \end{equation}
        où, les $p_i$ sont des polynômes en n. Il est importante de noter que contrairement
        à des suites arbitraires, les suites p-récursives, bien que comportant un nombre
        potentiellement infini de termes, peuvent être représentées exactement simplement
        avec la relation de récurrence, et les conditions initiales.
        Des exemples communs de suites p-récursives
        sont par exemple la suite de Fibonacci, ou la fonction factorielle.
        \begin{align*}
            \textnormal{Fibonacci : } F_{n+2} - F_{n+1} - F_n &= 0, \qquad F_0=0, F_1=1\\
            \textnormal{Factorielle : } (n+1)! - (n+1)u! &= 0, \qquad 0!=1
        \end{align*}
        De plus, les suites p-récursives forment un aneau ({\color{red} Donner les détails?}).
        Dès lors, il semble pertinent
        de réaliser une implémentation utilisant ces propriétés mathématiques afin de
        manipuler et utiliser les suites p-récursives.\\
        {\color{red} TODO : ore\_algebra what \& why\\ Intro ore...} Pour le présent projet,
        nous n'avons besoin que des notions traitant des suites p-récursives.
    \subsection{Python \& Sage}
        \label{ssec:sage}
        \par Sage est un outil de calcul formel libre.
        Il a été créé notamment pour proposer
        une alternative \textit{opensource} aux logiciels existants comme Mathematica,
        Matlab, Maple \ldots Contrairement à ces logiciels, Sage s'appuie sur des outils
        et librairies déjà existants comme NumPy, SciPy, matplotlib, FLINT et d'autres...
        L'utilisation de ces outils et unifiée et uniformisée au travers d'un langage
        basé sur Python. Ce langage présente une syntaxe qui diffère légérement de celle
        de Python. Ainsi, Sage est doté d'un "pré-analyseur",
        qui transforme les idiomes Sage en pur Python.
        Ainsi, il est possible d'écrire des librairies pour en Python pur ou en "langage sage".
        Bien qu'il existe également d'autres méthodes, on ne s'est intéressé qu'à celles-ci au
        cours du projet.
        \par Comme évoqué plus haut, Sage est basé sur Python, et c'est donc naturellement
        que nous avons choisi ce langage pour le projet.
        En particulier, Python 2, puisque Sage n'est pas compatible avec Python 3
        (bien que des efforts soient faits en ce sens).
        \par Bien que Sage fournisse de nombreuses librairies mathématiques,
        il n'inclut pas encore officiellement de librairie pour l'algèbre d'Ore.
        Nous avons eu donc recours à une bibliothèque en cours de développement par
        la communauté qui implémente l'algèbre d'Ore. L'utilisation de cette bibliothèque
        est présentée dans {\color{red} ref}.
        % TODO ref papier/site
    \subsection{La bibliotèque \textsc{OreAlgebra}}
\section{--Méthodologie de travail, et progression}
    \label{sec:methodo}
    \par La première tâche à laquelle nous nous sommes attelés a été une recherche bibliographique,
    pour comprendre le sujet (les suites p-récursives), et nos outils (Sage et Python).
    Les résultats de cette démarche sont présentés dans la partie \ref{sec:intro}.\\
    Puis nous avons commencé à discuter de l'implémentation. En accord avec notre encadrant,
    nous avons estimé qu'il était plus pertinent d'un point de vue pédagogique, de réaliser
    d'abord un module Python. Puis, une fois ce module eprouvé, si nous avions le temps,
    le réécrire avec la syntaxe de Sage. Cette manière de procédé devait permettre de se
    concentrer initialement sur le fond, et non la forme, puisque nous étions plus familier
    avec Python. {\color{blue} Il s'est avéré par la suite que nous n'avons pas eu le temps
    d'aborder la réécriture}.
    \subsection{--Module Python}
        \label{ssec:py}
        La base du module a été d'écrire une classe Python ({\color{red} init. n'étend aucun classes}). Cette classe devait notamment permettre
        d'utiliser la représentation basée sur la relation de récurrence, et des conditions initiales.
        Immédiatement après, nous avons surchargé l'opérateur \texttt{\uline getitem\uline } pour accéder
        au n-ième terme de la suite. Initialement, nous calculions tous les termes de la suite, jusqu'à 
        celui voulu, que nous renvoyions, mais cette méthode est très inefficace. Nous avons donc résolu
        d'utiliser la fonction \texttt{forward\_matrix} du module \textsc{ore\_algebra} à la place ({\color{red} exemple
        et comparaison complex avec Fibo?}) 
        \\{\color{red}calculer ts les elts vs calculer que le bon élement, exemple de différence tps exec sur 100000!})\\
        Par la suite, nous avons également surcharger les opérateurs d'addition, soustraction et multiplication, 
        en accord avec les lois de l'agèbre d'Ore.

\section{Implémentations}
    \subsection{Cacul d'un élément de la suite : \texttt{\uline getitem\uline } et \texttt{to\_list}}
        Pour calculer un élément d'une suite, il nous semblait logique de surcharger l'opérateur
        \texttt{\uline getitem\uline }, qui permet d'obtenir un élément de la manière suivante : 
        \texttt{u[42]}. En plus de ce choix, nous avons également surchargé la fonction
        \texttt{to\_list(n)}, qui nous permet d'obtenir tous les éléments de la suite, jusqu'à $n$.
        Nous avons utilisé plusieurs implémentations successives pour ces fonctions.
        Après être passés par l'innévitable \og méthode brouillone \fg, ou le code était duppliqué,
        et aucune des deux fonctions n'avaient une identité propre, nous avons finalement opté pour
        faire en sorte que \texttt{to\_list} appelle \texttt{\uline getitem\uline }. Une fois ce point
        éclairci, il nous restait à savoir comment implémenter le calcul même. Notre premier choix
        a été d'utiliser la fonction \texttt{to\_list} des objets \textsc{OreAlgebra}. En effet,
        pout un annihilateur et des conditions initiales données, celle-ci génère tous les éléments
        jusqu'à la valeur désirée.
        \par Cependant, nous avons remarqué que cette méthode présentait un défaut important : son
        temps de calcul. Ceci est dû au fait que pour calculer un élément $n$, cette fonction
        calcule tous les éléments dans l'intervalle $[0,n]$. Or ceci est complètement inneficace
        dans le cas où on ne veut que l'élément $n$. Pour palier à ce problème, notre encadrant nous 
        a suggéré d'utiliser la fonction \texttt{forward\_matrix\_bsplit}.
        {\color{red} Explications fctmt ou boite noire.}
        Cette fonction permet de calculer directement un élément de la suite, sans calculer tous ses
        prédécesseurs
        \footnote{Plus précisément, cette fonction calcule $k$ éléments, où  $k$ est l'ordre de la
        récurrence de la suite.}.
        \par Enfin, la dernière implémentation que nous ayons réalisé pour ce calcul est la suivante.
        Nous suspections que la fonction \texttt{forward\_matrix\_bsplit} était moins efficace que la
        fonction \texttt{to\_list} pour des valeurs basses de $n$. Ainsi, dans l'idée d'optimiser au
        mieux la fonction, nous avons comparé les exécutions de ces deux fonctions en faisant varier
        deux paramètres. D'une part nous avons fait varier $n$, et d'autre part l'ordre des suites
        utilisées. Les résultats sont présentés dans la figure \ref{fig:getitem}.
        \begin{figure} \begin{center}
            \includegraphics[scale=0.4]{figures/placeholder.png}
            \caption{\label{fig:getitem}\color{red}TODO}
        \end{center} \end{figure}
        {\color{red} TODO commentaires}. Ces résultats nous ont permis de déterminer empiriquement
        une valeur à partir de laquelle nous passons d'une méthode à l'autre.\\
        \par Le dernier point que nous avons implémenté concernant ces fonctions,
        est la possibilité d'utiliser les slices de Python. Les slices, sont des objets
        qui peuvent être paramètre de \texttt{\uline getitem\uline}. Pour une liste, ils permettent
        d'indiquer qu'on souhaite obtenir une \og tranche \fg\ de la liste, par exemple, les éléments
        8 à 13, éventuellement avec un pas. Dans notre cas, nous avons simplement adapté notre
        fonction, afin qu'elle calcule le premier élément de la tranche avec la méthode optimale,
        et les éléments suivants avec la méthode \texttt{to\_list} de l'annhilateur.

       

%----------------------------------------------------------------------------------------

\end{document}
